\documentclass[10pt,twocolumn, twoside, openright]{article}
\usepackage{fullpage}
% \usepackage{xymtex}
\usepackage[pdftex]{color}
\usepackage{indentfirst}
\usepackage{graphicx}
\usepackage[sort]{natbib}
\usepackage{amsmath}
\usepackage[pdftex, pdfpagemode={UseOutlines}, pdfauthor={Matt Richardson},bookmarks,
            pdfstartview ={Fit},pdftitle= {Study Guide for Preliminaries},
            colorlinks,hypertexnames,linkcolor={blue},citecolor ={blue}, urlcolor={red}]{hyperref}
            
%\usepackage{reaction,chemarrow}

%%\usepackage[sectionbib]{bibunits}
\usepackage{fancyhdr}
\setlength{\headheight}{12pt}
\pagestyle{fancy}


\usepackage{hyperref}

\setlength{\bibsep}{0ex}
\setlength{\parskip}{0.5ex}
\newcommand{\diff}{\mathrm{d}}
\newcommand{\diffc}[2]{\frac{\diff{#1}}{\diff{#2}}}

% Time based ordinary differential
\newcommand{\difft}[1]{\frac{\diff{#1}}{\diff{t}}}
\newcommand{\dg}{$\mathrm{^o}$}
\newcommand{\dgs}{$\mathrm{^o}$\hspace{1ex}}
\newcommand{\dgC}{$\mathrm{^o}$C}
\newcommand{\dgCs}{$\mathrm{^o}$C\hspace{1ex}}

\newcommand{\rxno}[2]{\mathrm{#1}&\rarrowfill{20pt}&\mathrm{#2}}
\newcommand{\erxno}[2]{\mathrm{#1}&\rightleftharpoonsfill{20pt}&\mathrm{#2}}
\newcommand{\erxn}[2]{\mathrm{#1}\rightleftharpoonsfill{20pt}\mathrm{#2}}
\newcommand{\rxn}[2]{\mathrm{#1}\rarrowfill{20pt}\mathrm{#2}}
% This command is for producing chemical concentrations 
% in the math environment
\newcommand{\conc}[1]{\mathrm{\left[{#1}\right]}}

%\defaultbibliography{f:/atmbib,f:/ice,f:/clouds,f:/wateruptake}
%\defaultbibliographystyle{agufull}
\setlength{\headsep}{12pt}

\begin{document}
%\chapter{
%
\section{Hardware}
\subsection{Photoacoustic Spectrometer}
%
\subsubsection{Housekeeping}
%
The PAS maintains an array of housekeeping data associated with each cell.  Each signal originates as an analog signal 
%
The thermistors not associated with the \href{http://www.teamwavelength.com/}{Wavelength} TECs (WCT3243) use a 10 V voltage divider with a 10 k$\Omega$ reference resistor ($R_{ref}$).  All thermistors in the system are flag mounted series \href{http://www.omega.com/toc_asp/frameset.html?book=Temperature&file=ON-930-44000&flag=1}{ON-930-44000} resistors which register a 10 k$\Omega$ resistance at 25\dg{C}.  The signal in is an anlog voltage.  The voltage is converted to a thermistor resistance $R_t$ using the following equation:
%
\begin{equation}\label{eq:Vdiv}
R_t = R_{ref}(\frac{V_source}{V_meas}-1)
\end{equation}
%
The thermistor resistance is then converted to a temperature using the Steinhart-Hart equation:
%
% Steinhart-Hart Equation
\begin{equation}\label{eq:SHH}
\frac{1}{T} = A+B\ln{R} + C(\ln{R})^3
\end{equation}
%
where $A$, $B$, and $C$ are the Steinhart-Hart coefficients.  The coefficients for the thermistor are defined by the thermistor model itself and can be solved for using the resistance as a function of temperature tables provided by the manufacturer; this results an a linear equation where $|\mathbf{1} \ln{\mathbf{R}} (\ln{\mathbf{R}})^3|\mathbf{X} = (\mathbf{T})^{-1}$ and $\mathbf{X}$ is a vector representing the coefficients.  For the series used in the PAS, the coefficients are 2.72e-3, 2.68e-4 and 6.17e-7 for $A$, $B$ and $C$ respectively. \par
%
An additional thermistor is provided for monitoring the temperature recorded by the Wavelength TECs.  The output of this TEC is filtered through the TEC electronics and is returned as a factor of the thermistor resistance.  So, if the thermistor is registering a temperature of 10\dg{C}, the voltage observed from the TEC eletronics will be 1.  This value is readily converted to a temperature using Eq. \ref{eq:SHH}.\par
%
Each cell in the PAS monitors the relative humidity $RH$ as well as the air temperture $T_{RH}$ using a Vaisala \href{http://www.fondriest.com/products/vaisala_hmp50yab1b1x.htm}{HMP50 (HMP50YCB1B1X)} humidity probe located near the outlet of the cell.  The probe provides two 0 to 5 V signals.  The conversion of these signals to real physical data is performed via the following two equations:
% Vaisala temperature
\begin{equation}\label{eq:TRH}
	T_{RH} = 20\times V -40
\end{equation}
%
and 
% Vaisala RH
\begin{equation}\label{eq:RH}
	RH = 20\times V
\end{equation}
%
The signals are enumerated in table \ref{tab:LFsig}.  All housekeeping signals are considered low frequency and are acquired as a rate of 1 kHz and averaged to 1 Hz.  In addition to the temperatures and relative humidities, the system also records the RMS laser power as seen by the photodiode at the rear of the cell.  This value is used to provide an indication of laser health and to normalize absorption signals recorded by the microphone (described below).
%
\begin{table*}[tb]
	\begin{tabular}{lccccc}
	\hline
	Signal & \multicolumn{2}{c}{Range} & Card & Channel & Scale\\
		& High & Low &&&\\
	\hline\hline
	Cell 1 Thermistor A & 0 & 5 & 6229 & ai0 & N/A\\
	Cell 2 Thermistor A & 0 & 5 & 6229 & ai5 & N/A\\
	Cell 3 Thermistor A & 0 & 5 & 6229 & ai10 & N/A\\
	Cell 4 Thermistor A & 0 & 5 & 6229 & ai16 & N/A\\
	Cell 5 Thermistor A & 0 & 5 & 6229 & ai21 & N/A\\
	Cell 1 Thermistor B & 0 & 5 & 6229 & ai1 & N/A\\
	Cell 2 Thermistor B & 0 & 5 & 6229 & ai6 & N/A\\
	Cell 3 Thermistor B & 0 & 5 & 6229 & ai11 & N/A\\
	Cell 4 Thermistor B & 0 & 5 & 6229 & ai17 & N/A\\
	Cell 5 Thermistor B & 0 & 5 & 6229 & ai22 & N/A\\
	Cell 1 TRH & 0 & 5 & 6229 & ai3 & \ref{eq:TRH}\\
	Cell 2 TRH & 0 & 5 & 6229 & ai8 & \ref{eq:TRH}\\
	Cell 3 TRH & 0 & 5 & 6229 & ai13 & \ref{eq:TRH}\\
	Cell 4 TRH & 0 & 5 & 6229 & ai19 & \ref{eq:TRH}\\
	Cell 5 TRH & 0 & 5 & 6229 & 24 & \ref{eq:TRH}\\
	Cell 1 RH & 0 & 5 & 6229 & ai2 & \ref{eq:RH}\\
	Cell 2 RH & 0 & 5 & 6229 & ai7 & \ref{eq:RH}\\
	Cell 3 RH & 0 & 5 & 6229 & ai12 & \ref{eq:RH}\\
	Cell 4 RH & 0 & 5 & 6229 & ai18 & \ref{eq:RH}\\
	Cell 5 RH & 0 & 5 & 6229 & ai23 & \ref{eq:RH}\\
	Cell 1 LRMS & 0 & 5 & 6229 & ai4 & N/A\\
	Cell 2 LRMS & 0 & 5 & 6229 & ai9 & N/A\\
	Cell 3 LRMS & 0 & 5 & 6229 & ai14 & N/A\\
	Cell 4 LRMS & 0 & 5 & 6229 & ai20 & N/A\\
	Cell 5 LRMS & 0 & 5 & 6229 & ai25 & N/A\\
	\hline
	\end{tabular}
	\caption{Low frequency signal layout.  Scales specified above are applied through the DAQmx functionality.  Thermistors are scaled in the software.  \textbf{LRMS} refers to the laser RMS and this value is not scaled.  The laser RMS provides an indication of the laser power that the photodiodes are seeing (which should be minimally effected by interference of aerosol with the beam).}
	\label{tab:LFSig}
\end{table*}
%
\subsubsection{Laser IO}
%

%
\begin{table*}[tb]
	\centering
		\begin{tabular}{lcccc}
		\hline
		Signal & \multicolumn{2}{c}{Range} & Card & Channel\\
		& High & Low &&\\
			\hline\hline
			Cell 1 Microphone & 5 & -5 & 6259 & ai0\\
			Cell 2 Microphone & 5 & -5 & 6259 & ai2\\
			Cell 3 Microphone & 5 & -5 & 6259 & ai4\\
			Cell 4 Microphone & 5 & -5 & 6259 & ai6\\
			Cell 5 Microphone & 5 & -5 & 6259 & ai16\\
			Cell 1 Photodiode & 5 & -5 & 6259 & ai1\\
			Cell 2 Photodiode & 5 & -5 & 6259 & ai3\\
			Cell 3 Photodiode & 5 & -5 & 6259 & ai5\\
			Cell 4 Photodiode & 5 & -5 & 6259 & ai7\\
			Cell 5 Photodiode & 5 & -5 & 6259 & ai17\\
			\hline
		\end{tabular}
	\caption{High frequency signal layout.}
	\label{tab:HFsig}
\end{table*}
\subsection{Cavity Ring Down Spectrometer}

\section{Software Overview}
%
Most software described herein is object-oriented.  All instruments are built off of a single parent called the \textbf{Instrument} class.\par
%
\section{Instrument Class}
%
The Instrument class is the top-level class for most instrument acquisition applications.  The Instrument class is, for all purposes, a virtual class which only functions after instantiation.  The Instrument class is a simple class which contains a set of must-override routines as well as some static VIs that provide soem generic functionality, particularly if the class will be used in a state-machine setup. \par
%
Each instrument is assumed to do the following:
%
\begin{itemize}
\item \textbf{Configure} at startup.  The process of configuring an instrument is specific to the implementation, but all instruments require it in order to make measurements, return data, produce signals, etc.  Some instruments will have a configuration file that may be read in at startup while others may simply need to specify pre-defined tasks and scales.
\item {Get Data} during operation.  This may require communication with serial ports or calling DAQmx routines to read data off of data acquisition cards.
\item {Analyze Data} at some point during operation.  While this function may be combined with the \textbf{Get Data}
\end{itemize}
%
\section{Photoacoustic Spectrometer}
%
Every photoacoustic spectrometer (PAS) encompasses only two things:
\begin{itemize}
	\item A cell through in which the bean and aerosol interact.
	\item The data produced by this interaction.
\end{itemize}

\section{Cavity Ring Down}
%
The CRDS fires three different lasers at a rate of 1000 shots per second into eight different cells.  Each shot results in a single decay on each channel.  Data is collected at a rate up to 2.5 mega-samples a second using a PXI S-series 6133 from the point of the shot being fired until the smaller of 500 $\mu$s or 5x the previous decay constant $\tau$.  The data is aggregated on a 1 second basis and then the 1000 individual traces on each channel are analyzed using a routine defined by \citet{}. \par
\subsection{Fitting Taus}
%

\end{document}